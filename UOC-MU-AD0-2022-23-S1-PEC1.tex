% Options for packages loaded elsewhere
\PassOptionsToPackage{unicode}{hyperref}
\PassOptionsToPackage{hyphens}{url}
\PassOptionsToPackage{dvipsnames,svgnames,x11names}{xcolor}
%
\documentclass[
  letterpaper,
  DIV=11,
  numbers=noendperiod]{scrartcl}

\usepackage{amsmath,amssymb}
\usepackage{lmodern}
\usepackage{iftex}
\ifPDFTeX
  \usepackage[T1]{fontenc}
  \usepackage[utf8]{inputenc}
  \usepackage{textcomp} % provide euro and other symbols
\else % if luatex or xetex
  \usepackage{unicode-math}
  \defaultfontfeatures{Scale=MatchLowercase}
  \defaultfontfeatures[\rmfamily]{Ligatures=TeX,Scale=1}
\fi
% Use upquote if available, for straight quotes in verbatim environments
\IfFileExists{upquote.sty}{\usepackage{upquote}}{}
\IfFileExists{microtype.sty}{% use microtype if available
  \usepackage[]{microtype}
  \UseMicrotypeSet[protrusion]{basicmath} % disable protrusion for tt fonts
}{}
\makeatletter
\@ifundefined{KOMAClassName}{% if non-KOMA class
  \IfFileExists{parskip.sty}{%
    \usepackage{parskip}
  }{% else
    \setlength{\parindent}{0pt}
    \setlength{\parskip}{6pt plus 2pt minus 1pt}}
}{% if KOMA class
  \KOMAoptions{parskip=half}}
\makeatother
\usepackage{xcolor}
\setlength{\emergencystretch}{3em} % prevent overfull lines
\setcounter{secnumdepth}{-\maxdimen} % remove section numbering
% Make \paragraph and \subparagraph free-standing
\ifx\paragraph\undefined\else
  \let\oldparagraph\paragraph
  \renewcommand{\paragraph}[1]{\oldparagraph{#1}\mbox{}}
\fi
\ifx\subparagraph\undefined\else
  \let\oldsubparagraph\subparagraph
  \renewcommand{\subparagraph}[1]{\oldsubparagraph{#1}\mbox{}}
\fi


\providecommand{\tightlist}{%
  \setlength{\itemsep}{0pt}\setlength{\parskip}{0pt}}\usepackage{longtable,booktabs,array}
\usepackage{calc} % for calculating minipage widths
% Correct order of tables after \paragraph or \subparagraph
\usepackage{etoolbox}
\makeatletter
\patchcmd\longtable{\par}{\if@noskipsec\mbox{}\fi\par}{}{}
\makeatother
% Allow footnotes in longtable head/foot
\IfFileExists{footnotehyper.sty}{\usepackage{footnotehyper}}{\usepackage{footnote}}
\makesavenoteenv{longtable}
\usepackage{graphicx}
\makeatletter
\def\maxwidth{\ifdim\Gin@nat@width>\linewidth\linewidth\else\Gin@nat@width\fi}
\def\maxheight{\ifdim\Gin@nat@height>\textheight\textheight\else\Gin@nat@height\fi}
\makeatother
% Scale images if necessary, so that they will not overflow the page
% margins by default, and it is still possible to overwrite the defaults
% using explicit options in \includegraphics[width, height, ...]{}
\setkeys{Gin}{width=\maxwidth,height=\maxheight,keepaspectratio}
% Set default figure placement to htbp
\makeatletter
\def\fps@figure{htbp}
\makeatother

\KOMAoption{captions}{tableheading}
\makeatletter
\makeatother
\makeatletter
\makeatother
\makeatletter
\@ifpackageloaded{caption}{}{\usepackage{caption}}
\AtBeginDocument{%
\ifdefined\contentsname
  \renewcommand*\contentsname{Table of contents}
\else
  \newcommand\contentsname{Table of contents}
\fi
\ifdefined\listfigurename
  \renewcommand*\listfigurename{List of Figures}
\else
  \newcommand\listfigurename{List of Figures}
\fi
\ifdefined\listtablename
  \renewcommand*\listtablename{List of Tables}
\else
  \newcommand\listtablename{List of Tables}
\fi
\ifdefined\figurename
  \renewcommand*\figurename{Figure}
\else
  \newcommand\figurename{Figure}
\fi
\ifdefined\tablename
  \renewcommand*\tablename{Table}
\else
  \newcommand\tablename{Table}
\fi
}
\@ifpackageloaded{float}{}{\usepackage{float}}
\floatstyle{ruled}
\@ifundefined{c@chapter}{\newfloat{codelisting}{h}{lop}}{\newfloat{codelisting}{h}{lop}[chapter]}
\floatname{codelisting}{Listing}
\newcommand*\listoflistings{\listof{codelisting}{List of Listings}}
\makeatother
\makeatletter
\@ifpackageloaded{caption}{}{\usepackage{caption}}
\@ifpackageloaded{subcaption}{}{\usepackage{subcaption}}
\makeatother
\makeatletter
\@ifpackageloaded{tcolorbox}{}{\usepackage[many]{tcolorbox}}
\makeatother
\makeatletter
\@ifundefined{shadecolor}{\definecolor{shadecolor}{rgb}{.97, .97, .97}}
\makeatother
\makeatletter
\makeatother
\ifLuaTeX
  \usepackage{selnolig}  % disable illegal ligatures
\fi
\IfFileExists{bookmark.sty}{\usepackage{bookmark}}{\usepackage{hyperref}}
\IfFileExists{xurl.sty}{\usepackage{xurl}}{} % add URL line breaks if available
\urlstyle{same} % disable monospaced font for URLs
\hypersetup{
  colorlinks=true,
  linkcolor={blue},
  filecolor={Maroon},
  citecolor={Blue},
  urlcolor={Blue},
  pdfcreator={LaTeX via pandoc}}

\title{\includegraphics[width=5in,height=\textheight]{images/uoc_masterbrand_2linies_positiu.png}

Análisis de datos Ómicos (M0-157) Primera prueba de evaluación
continua.}
\author{}
\date{}

\begin{document}
\maketitle
\ifdefined\Shaded\renewenvironment{Shaded}{\begin{tcolorbox}[boxrule=0pt, frame hidden, borderline west={3pt}{0pt}{shadecolor}, enhanced, interior hidden, sharp corners, breakable]}{\end{tcolorbox}}\fi

\textbf{Fecha de publicación del enunciado: 31/10/2022}

\textbf{Fecha límite para realizar aportaciones: 13/11/2022}\footnote{La
  fecha de entrega es la que se indica en el enunciado de la PEC. En
  caso de no coincidir con la indicada en el aula, ésta será la que
  predomine.}

\hypertarget{presentaciuxf3n-y-objetivos}{%
\subsection{Presentación y
objetivos}\label{presentaciuxf3n-y-objetivos}}

Esta PEC completa la introducción a las ómicas mediante un ejercicio de
repaso y ampliación que nos permite introducirnos a una de las
herramientas que trabajaremos más en este curso, el conjunto de paquetes
para análisis de datos ómicos conocido como Bioconductor.

No tenéis que pensar en la PEC como algo que haréis tras completar los
tres primeros módulos sino como unos ejercicios que os servirán, sí,
para repasar y para ser evaluados, pero también para aprender conceptos
nuevos que se han introducido, pero no practicado antes.

Para poder entender y llevar a cabo esta primera parte, tenéis que
familiarizaros con \href{http://bioconductor.org}{Bioconductor} y con el
aquete Biobase, en el que se implementan los contenedores de datos
ómicos denominados expressionSets. Para ello, una vez leída la
presentación del campus podéis clonar y seguir el caso de uso:
"\url{https://github.com/ASPteaching/Omics_Data_Analysis-Case_Study_0-Introduction_to_BioC}".
Mi consejo es que \textbf{hagáis esto antes de empezar la PEC.}

\hypertarget{descripciuxf3n-de-la-pec}{%
\subsection{Descripción de la PEC}\label{descripciuxf3n-de-la-pec}}

El objetivo de esta PEC es que seleccionéis un estudio de microarrays de
la lista disponible en el archivo adjunto "GEOdatasets.xls" y llevéis a
cabo un estudio exploratorio \emph{similar} al que se hacia en el caso
de estudio del primer debate que, recordad, estaba disponible en :
\url{https://github.com/ASPteaching/Analisis_de_datos_omicos-Ejemplo_0-Microarrays}

En la práctica esto significa que debéis:

\begin{itemize}
\item
  Leer los datos
\item
  Determinar su estructura y el diseño del estudio que los ha generado
\item
  Realizar un análisis exploratorio univariante y multivariante similar
  al del caso de estudio.
\end{itemize}

\hypertarget{cambios-respecto-del-caso-de-estudio}{%
\subsubsection{Cambios respecto del caso de
estudio}\label{cambios-respecto-del-caso-de-estudio}}

\begin{itemize}
\item
  En el ejemplo del primer debate, además de la exploración
  multivariante, se lleva a cabo una selección de genes. \textbf{Aquí no
  tenéis que hacerlo}, porque ésta precisamente será uno de los
  objetivos de la siguiente PEC
\item
  La lectura y el manejo de los datos del ejemplo se lleva a cabo usando
  funciones básicas de R. El objetivo de la PEC és que \textbf{no lo
  hagáis así} sino utilizando funciones de Bioconductor. En concreto:

  \begin{itemize}
  \item
    Debéis utilizar la función \texttt{getGEO} del paquete
    \texttt{GEOquery} para descargar los datos.
  \item
    Debéis utilizar las funciones \texttt{exprs} y/o \texttt{pData} para
    acceder a los datos y los metadatos.
  \end{itemize}

  Aunque esto no tenga sentido en estos momentos lo tendrá una vez
  hayáis trabajado con la introducción a Bioconductor del caso.
\end{itemize}



\end{document}
